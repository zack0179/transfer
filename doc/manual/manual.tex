\documentclass[floatfix,aps,prd,nofootinbib,superscriptaddress,preprint]{revtex4}

\pdfoutput=1
\usepackage{graphicx}
\usepackage{amsmath}
\usepackage{amssymb}
\usepackage{slashed}
\usepackage{mathtools}
\usepackage[utf8]{inputenc}
\usepackage[colorlinks=true,linkcolor=blue,bookmarksopen,bookmarksnumbered]{hyperref}
\usepackage{color}
\usepackage[usenames,dvipsnames]{xcolor}
\usepackage[normalem]{ulem}
\usepackage{soul}
\usepackage{units}
\usepackage{rotating}
\usepackage{hhline,multirow,tabularx}
\usepackage{bm}
\usepackage{algorithm}% http://ctan.org/pkg/algorithm
\usepackage{algpseudocode}% http://ctan.org/pkg/algorithmicx

% math macros 
\newcommand{\T}[1]{\boldsymbol{#1}_{\text{T}}}
\newcommand{\kT}{\ensuremath{k_{\rm T}}}
\newcommand{\ktmax}{\ensuremath{k_{\rm T max}}}
\newcommand{\ktmaxsq}{\ensuremath{k_{\rm T max}^2}}
\newcommand\3[1]{\boldsymbol{#1}}
\newcommand{\Tsc}[2]{#1_{#2\text{T}}}
\newcommand{\Tscsq}[2]{#1^2_{#2\text{T}}}
\newcommand{\no}{\nonumber \\}
\newcommand{\parz}[1]{\ensuremath{\left(#1\right)}}
\newcommand{\order}[1]{\ensuremath{O\parz{#1}}}
\newcommand{\mhad}{\ensuremath{M}}
\newcommand{\alfa}{\ensuremath{\alpha}}
\newcommand{\xbj}{\ensuremath{x_{\rm bj}}}
\newcommand{\xn}{\ensuremath{x_{\rm n}}}
\newcommand{\mquark}{\ensuremath{m_{q}}}
\newcommand{\mgluon}{\ensuremath{m_{s}}}
\newcommand{\spectator}{\ensuremath{p_s}}
\newcommand{\jet}{\ensuremath{p_q}}
\newcommand{\pdfblob}{{\ensuremath{\mathcal F}}}
\newcommand{\ffblob}{{\ensuremath{\mathcal J}}}
\newcommand{\prop}{{[ \ensuremath{\rm Prop} ]}}
\newcommand{\num}{{\ensuremath{\rm T}}}
\newcommand{\jac}[1]{{\ensuremath{[ {\rm Jac_{#1}} ] }}}
\newcommand{\arrowcom}[1]{\textcolor{red}{\textbf{$\Longrightarrow$ #1}} \\}
\newcommand{\arrowcomtwo}[1]{\textcolor{blue}{ \textbf{$\Longrightarrow$ #1}} \\}
\newcommand{\arrowcomthree}[1]{\textcolor{magenta}{ \textbf{$\Longrightarrow$ #1}} \\}
\newcommand{\diff}[1]{\mathrm{d}#1}
\newcommand{\moderate}{{\color{red}low}}

\newcommand{\vect}[1]{\ensuremath{{\bm{#1}}}}
\newcommand{\bfkperp}{{\bf k}_\perp} 
\newcommand{\bfpperp}{{\bf P}_\perp} 
\newcommand{\bfhp}{\hat{\bf h}} 
\newcommand{\bfPhperp}{{\bf P}_{hT}}
\newcommand{\Phperp}{P_{hT}}
\newcommand{\kperp}{k_\perp}
\newcommand{\pperp}{P_\perp}
\newcommand{\avkperp}{\la \kperp^2 \ra}
\newcommand{\avpperp}{\la \pperp^2 \ra}
\newcommand{\la}{\langle}
\newcommand{\ra}{\rangle}


% ref macros
\newcommand{\eref}[1]{Eq.~(\ref{e.#1})}
\newcommand{\erefs}[2]{Eqs.~(\ref{e.#1})--(\ref{e.#2})}
\newcommand{\fref}[1]{Fig.~\ref{f.#1}}
\newcommand{\frefs}[2]{Figs.~\ref{f.#1}--\ref{f.#2}}
\newcommand{\aref}[1]{Appendix~\ref{a.#1}}
\newcommand{\sref}[1]{Sec.~\ref{s.#1}}
\newcommand{\ssref}[1]{Section~\ref{ss.#1}}
\newcommand{\sssref}[1]{Section~\ref{sss.#1}}
\newcommand{\tref}[1]{Table~\ref{t.#1}}








\begin{document}\title{Extraction and Validation Framework (EVA)}
{\color{red} TODO: update authors and affiliations}
\author{A.~Prokudin}
\author{D.~Riser}
\author{N.~Sato}
\author{K.~Tezgin}
\email{nsato@jlab.org}
\email{kemal.tezgin@uconn.edu}
\email{david.riser@uconn.edu}
\affiliation{University of Conneticut}

\maketitle

\newpage

\newcommand{\avp}{\ensuremath{\la P_{\perp}^2\ra_q}}
\newcommand{\avk}{\ensuremath{\la k_{\perp}^2\ra_q}}

\newcommand{\pdf}{{\color{blue} \ensuremath{f_1^q}}}
\newcommand{\ff}{{\color{red} \ensuremath{D_1^{q}}}}
\newcommand{\ppdf}{{\color{Emerald} \ensuremath{g_1^q}}}
\newcommand{\transversity}{{\color{gray} \ensuremath{h_1^q}}}
\newcommand{\sivers}{{\color{brown} \ensuremath{f^{\perp(1)q}_{1T}}}}
\newcommand{\boermulders}{{\color{purple} \ensuremath{h^{\perp(1)q}_1 }}}
\newcommand{\pretzelosity}{{\color{violet} \ensuremath{h^{\perp(2)q}_{1T} }}}
\newcommand{\collins}{{\color{orange} \ensuremath{H^{\perp(1)q}_{1} }}}
\newcommand{\gww}{{\color{magenta} \ensuremath{g^{\perp (1) q}_{1T}}}}
\newcommand{\gt}{{\color{magenta} \ensuremath{g^{q}_{T}}}}
\newcommand{\hww}{{\color{cyan}\ensuremath{h^{\perp (1) q}_{1L}}}}
\newcommand{\gtperp}{{\color{violet} \ensuremath{g^{\perp(2)q}_{T} }}}
\newcommand{\fperp}{{\color{cyan}\ensuremath{f^{\perp (1) q}}}}
\newcommand{\htperp}{{\color{cyan}\ensuremath{h^{\perp (1) q}_T}}}
\newcommand{\htt}{{\color{cyan}\ensuremath{h^{(1) q}_T}}}
\newcommand{\ftperp}{{\color{violet} \ensuremath{f^{\perp(2)q}_{T} }}}

\newcommand{\wpdf}{0.57}
\newcommand{\wff}{0.12}
\newcommand{\wppdf}{0.19}
\newcommand{\wtransversity}{0.25}
\newcommand{\wsivers}{0.143}
\newcommand{\wboermulders}{0.085}
\newcommand{\wpretzelosity}{0.137}
\newcommand{\wcollins}{0.11}
\newcommand{\wgww}{0.19}
\newcommand{\whww}{0.25}


\section{SIDIS setup}

The differential cross section is given by 

\begin{align}
\frac{d\sigma}{dx ~ dy ~ d\phi_S ~ dz ~ d\phi_h ~ d\Phperp^2} = 
\frac{\alpha^2}{xyQ^2}\frac{y^2}{2(1-\varepsilon)}\left(1+\frac{\gamma^2}{2x} \right) 
\sum_{i=1}^{18} F_i(x,z,Q^2,\Phperp^2) \beta_i 
\label{e.xsec}
\end{align}

\begin{table}[h!]
\begin{tabular}{|c|c|c|c|c|}
\hline
$F_i$    & Standard label                    & $\beta_i$\\\hline
$F_1$    & $F_{UU,T}$                        & 1\\\hline
$F_2$    & $F_{UU,L}$                        & $\varepsilon$\\\hline
$F_3$    & $F_{LL}$                          & $S_{||} \lambda_e \sqrt{1-\varepsilon^2}$\\\hline
$F_4$ & $F_{UT}^{\sin(\phi_h+\phi_S)}$    & $|\vec{S}_\perp|\varepsilon\sin(\phi_h+\phi_S)$\\\hline
$F_5$ & $F_{UT,T}^{\sin(\phi_h-\phi_S)}$  & $|\vec{S}_\perp|\sin(\phi_h-\phi_S)$\\\hline
$F_6$ & $F_{UT,L}^{\sin(\phi_h-\phi_S)}$  & $|\vec{S}_\perp|\varepsilon\sin(\phi_h-\phi_S)$\\\hline
$F_7$    & $F_{UU}^{\cos 2\phi_h}$           & $\varepsilon ~ \cos(2\phi_h)$\\\hline
$F_8$ & $F_{UT}^{\sin(3\phi_h - \phi_S)}$ & $|\vec{S}_\perp| \varepsilon \sin(3\phi_h - \phi_S)$\\\hline
$F_9$ & $F_{LT}^{\cos(\phi_h-\phi_S)}$    & $|\vec{S}_\perp|\lambda_e \sqrt{1-\varepsilon^2}\cos(\phi_h-\phi_S)$\\\hline
$F_{10}$    & $F_{UL}^{\sin ~ 2\phi_h}$         & $S_{||}\varepsilon \sin(2\phi_h)$\\\hline
$F_{11}$ & $F_{LT}^{ \cos\phi_S}$            & $|\vec{S}_\perp|\lambda_e \sqrt{2\varepsilon(1-\varepsilon)} \cos\phi_S$\\\hline
$F_{12}$    & $F_{LL}^{\cos\phi_h}$             & $S_{||} \lambda_e  \sqrt{2\varepsilon(1-\varepsilon)} \cos\phi_h$\\\hline
$F_{13}$ & $F_{LT}^{\cos(2\phi_h-\phi_S)}$   & $|\vec{S}_\perp|\lambda_e \sqrt{2\varepsilon(1-\varepsilon)} \cos(2\phi_h-\phi_S)$\\\hline
$F_{14}$    & $F_{UL}^{\sin\phi_h}$             & $S_{||}\sqrt{2\varepsilon (1+\varepsilon)}\sin\phi_h$\\\hline
$F_{15}$    & $F_{LU}^{\sin\phi_h}$             & $\lambda_e\sqrt{2\varepsilon (1-\varepsilon)} \sin\phi_h $\\\hline
$F_{16}$    & $F_{UU}^{\cos \phi_h}$            & $\sqrt{2\varepsilon (1+\varepsilon)} ~ \cos\phi_h$\\\hline
$F_{17}$ & $F_{UT}^{\sin\phi_S}$             & $|\vec{S}_\perp| \sqrt{2\varepsilon (1+\varepsilon)}\sin\phi_S$\\\hline
$F_{18}$ & $F_{UT}^{\sin(2\phi_h-\phi_S)}$   & $|\vec{S}_\perp| \sqrt{2\varepsilon (1+\varepsilon)}\sin(2\phi_h-\phi_S)$\\\hline
\end{tabular}
\end{table}
where
\begin{align}
\varepsilon &= \frac{1-y-\frac{1}{4}\gamma^2 y^2}{1-y+\frac{1}{2} y^2 + \frac{1}{4}\gamma^2 y^2}\, , \\
\gamma &=  \frac{2 M x}{Q} \, ,
\label{e.epsilon}
\end{align}
and in the nucleon rest frame the polarization vector is given by 
$S=(0,{\bm S}_\perp,S_L)$ with ${\bm S}_\perp^2+S_L^2=1$.

The 18 structure function in SIDIS at leading-order will be expressed
in the context of WW-type approximation in terms of a ``minimal''
TMD basis using the gaussian ansatz:
%
\begin{align}
{\cal F}_q(\xi,p_{\perp})&={\cal K}_q ~ {\cal C}_q(\xi) \frac{\exp\left(-k_{\perp}^2/\omega_q\right)}{\pi \omega_q}\\
{\cal D}_q(\xi,p_{\perp})&={\cal K}_q ~ {\cal C}_q(\xi) \frac{\exp\left(-P_{\perp}^2/\omega_q\right)}{\pi \omega_q}.
\end{align}
%
We denote the transverse momentum of the quark inside a fast moving
proton by $\bfkperp$. We use the notation $\bfpperp$ for the
transverse momentum of the quark relative to the fragmenting parton
motion. The structure functions are expressed as
%
\begin{align}
F&=\sum_q e_q^2 ~ {\cal K}_q ~ {\cal F}_q(x) ~{\cal D}_q(z) \frac{\exp\left(-\Phperp^2/\Omega_q\right)}{\pi \Omega_q}\\
\Omega_q&=z^2\avk + \avp
\end{align}

%

\begin{table}[h!]
\begin{tabular}{|c|c|c|c|c|}
\hline
type       & Name         & ${\cal K}_q$                  & ${\cal C}_q$    \\\hline
${\cal F}_q$ & upol. PDF    & $1$                           & $\pdf$          \\\hline
${\cal F}_q$ & pol. PDF     & $1$                           & $\ppdf$         \\\hline
${\cal F}_q$ & Transversity & $1$                           & $\transversity$ \\\hline
${\cal F}_q$ & Sivers       & $\frac{2M^2}{\omega_q}$       & $\sivers$       \\\hline
${\cal F}_q$ & Boer-Mulders & $\frac{2M^2}{\omega_q}$       & $\boermulders$  \\\hline
${\cal F}_q$ & Pretzelosity & $\frac{2M^4}{\omega_q^2}$       & $\pretzelosity$ \\\hline
${\cal F}_q$ & Worm Gear    & $\color{red}\frac{2M^2}{\omega_q}$                           & $\gww$          \\\hline
${\cal F}_q$ & Worm Gear    & $\color{red}\frac{2M^2}{\omega_q}$                           & $\hww$          \\\hline
${\cal F}_q$ & twist-3    & $1$                           & $\gt$          \\\hline
${\cal F}_q$ & twist-3  & $\frac{2M^4}{\omega_q^2}$       & $\gtperp$ \\\hline
${\cal F}_q$ & twist-3  & $\frac{2M^2}{\omega_q}$       & $\fperp$ \\\hline
${\cal F}_q$ & twist-3  & $\frac{2M^2}{\omega_q}$       & $\htperp$ \\\hline
${\cal F}_q$ & twist-3  & $\frac{2M^2}{\omega_q}$       & $\htt$ \\\hline
${\cal F}_q$ & twist-3  & $\frac{2M^4}{\omega_q^2}$       & $\ftperp$ \\\hline
\hline
${\cal C}_q$ & FF           & $1$                           & $\ff$           \\\hline
${\cal C}_q$ & Collins      & $\frac{2z^2 m_h^2}{\omega_q}$ & $\collins$      \\\hline
\end{tabular}
\caption{The minimal basis}
\label{t.mbasis}
\end{table}


\begin{table}[h!]
\begin{tabular}{|c|c|c|c|c|c|c|}
\hline
         &                                     & ${\cal{K}}_q$                                                                               & ${\cal F}_q(x)$ & ${\cal D}_q(z)$ \\\hline
$F_1$    & $F_{UU,T}$                          & $x$                                                                                         & $\pdf$          & $\ff $          \\\hline
$F_2$    & $F_{UU,L}$                          & $0$                                                                                         &                 & \\\hline
$F_3$    & $F_{LL}$                            & $x$                                                                                         & $\ppdf$         & $\ff$           \\\hline
$F_4$    & $F_{UT}^{\sin(\phi_h+\phi_S)}$      & $\frac{2 x z \Phperp  m_h}{w_q}$                                                            & $\transversity$ & $\collins$      \\\hline
$F_5$    & $F_{UT,T}^{\sin(\phi_h-\phi_S)}$    & $-\frac{2xzM \Phperp}{w_q}$                                                                 & $\sivers$       & $\ff$           \\\hline
$F_6$    & $F_{UT,L}^{\sin(\phi_h-\phi_S)}$    & $0$                                                                                         &                 & \\\hline
$F_7$    & $F_{UU}^{\cos(2\phi_h)}$            & $\frac{4 x z^2 M \Phperp^2 m_h}{w_q^2} $                                                    & $\boermulders$  & $\collins $     \\\hline
$F_8$    & $F_{UT}^{\sin(3\phi_h-\phi_S)}$     & $\frac{2 x z^3 \Phperp^3 m_h M^2}{w_q^3} $                                                 & $\pretzelosity$ & $\collins $     \\\hline
$F_9$    & $F_{LT}^{\cos(\phi_h -\phi_S)}$     & $ \frac{2xzM\Phperp}{w_q} $                                                                 & $\gww$          & $\ff$           \\\hline
$F_{10}$ & $F_{UL}^{\sin(2\phi_h)}$            & $ \frac{4 xz^2M \Phperp^2 m_h}{w_q^2} $                                                     & $\hww$          & $\collins $     \\\hline
$F_{11}$ & $F_{LT}^{\cos\phi_S}$               & $\color{red} -\frac{2M}{Q} x$           & $\gt$          & $\ff$           \\\hline
$F_{12}$ & $F_{LL}^{\cos\phi_h}$               & $-\frac{2xz\Phperp}{Q}\frac{\avk}{w_q}$                                                     & $\ppdf$         & $\ff$           \\\hline
$F_{13}$ & $F_{LT}^{\cos(2\phi_h -\phi_S)}$    & $\color{red} -\frac{2 x z^2 M^3 \Phperp^2}{Q}\frac{1}{w_q^2}$                                          & $x\gtperp$          & $\ff$           \\\hline
$F_{14}$ & $F_{UL}^{\sin\phi_h}$               & $\color{red} \frac{4 M m_h}{Q}  \frac{\Phperp z}{w_q}$                        & $\hww$          & $\collins$      \\\hline
$F_{15}$ & $F_{LU}^{\sin\phi_h}$               & $0$                                                                                         &                 & \\\hline
$F_{16}$ & $F_{UU}^{\cos\phi_h}(i)$            & {\color{red}0} & $\boermulders$  & $\collins$      \\\hline
$F_{16}$ & $F_{UU}^{\cos\phi_h}(ii)$           & $\color{red} -\frac{2M}{Q} \frac{2 xz\Phperp M}{w_q} $                                                           & $x\fperp$          & $\ff $          \\\hline
$F_{17}$ & $F_{UT}^{\sin\phi_S}(i)$            & ${\color{red}0} $                & $\sivers$       & $\ff$           \\\hline
$F_{17}$ & $F_{UT}^{\sin\phi_S}(ii)$           & $\color{red} \frac{2M}{Q} \frac{4xz^2 m_h M}{w_q}(1-\frac{\Phperp^2}{w_q}) $                       & $-\frac{1}{2}x(\htperp-\htt)$ & $\collins$      \\\hline
$F_{18}$ & $F_{UT}^{\sin(2\phi_h-\phi_S)}(i)$  & $\color{red} \frac{2M}{Q} x \frac{M^2 z^2 \Phperp^2}{w_q^2} $                                                   & $x\ftperp$       & $\ff$           \\\hline
$F_{18}$ & $F_{UT}^{\sin(2\phi_h-\phi_S)}(ii)$ & $\color{red}-\frac{2M^2}{Q} x \frac{4z^2 \Phperp^2 M m_h}{w_q^2}$                                        & $-\frac{1}{2}x(\htperp+\htt)$ & $\collins$      \\\hline
$F_{19}$ & $F_{CAHN}^{\cos(2\phi_h)}$          & $\frac{1}{Q^2} \frac{2 x z^2 \Phperp^2 \avk^2}{w_q^2}$                                      & $\pdf$          & $\ff $          \\\hline
\end{tabular}
\caption{SIDIS structure functions up to twist 3}
\label{t.chi2}
\end{table}
twist-3 functions can be related to twist-2 functions using the following formulas:
\begin{subequations} 
\begin{equation}
\gt(x) = \int_x^1\frac{ d y}{y}\,\ppdf(y) 
\end{equation}
\begin{equation}
	x\gtperp(x) = \frac{\la\kperp^2\ra_{\gww}}{M^2}\;
	\gww(x)\,,
\end{equation}
\begin{equation}
	x\,\fperp(x) = \frac{\la\kperp^2\ra_{\pdf}}{2M^2}\,\pdf(x)
	\, , 
\end{equation}
\begin{equation}
	-\frac{1}{2}x(\htperp(x)-\htt(x)) = \frac{\la\kperp^2\ra_{\transversity}}{2M^2}\,\transversity(x)
	\, , 
\end{equation}
\begin{equation}
	-\frac{1}{2}x(\htperp(x)+\htt(x)) =  \pretzelosity(x)
	\, , 
\end{equation}
\end{subequations}

 

\newpage
\section{SIA setup}

{\color{red} TODO: add table of SIA structure functions}

\newpage
\section{Tutorial}
{\color{red} TODO: Add more details }

Here is a general workflow

\begin{itemize}
\item Identify which parameters are needed to be fitted for a given set of TMDs
\item Identify available observables to extract the TMDs 
\item Create xlsx files for each data sets (see the database repo)
\item Create and input file (see i.e. fitlab/inputs/upol.py). Here you will specify 
      which data sets are going to be used
\item Test in a single fit (fitter.py) that a reasonable $\chi^2$ is obtained 
\item Once the test fit is ready, proceed to run a multinest which is an nested sampling algorithm 
      to map out the likelihood function. In the input file you must specify the output.
      Look for the line "output dir " in the input file. It is recommended to store the results in the 
      repo analysis. 
\item Once the multinest finishes it produces a nestout file at the path that was specified. 
      Use one of the jupyter templates to proceed to analyze the output. You need to place the notebook 
      at the root of analysis. Otherwise the paths wont match.  
\end{itemize}

If you want to run the codes but you want to modify things here and there, 
create a new workspace via pacman/gen-fitpack, and cd to fitpack. The latter 
has an exact copy of all the repos, and you can do with that whatever you want. 
Once you see that some of your modifications should be placed in the actual
repositories, proceed to do it.

\newpage
\section{Unpolarized TMDS}

\begin{itemize}
\item {\color{red} TODO: table of data sets with the following columns: exp, observable, num points, chi2 }
\item {\color{red} TODO: what TMDS are going to be fitted? what is the parametrization}
\item {\color{red} TODO: distribution of the parameters}
\item {\color{red} TODO: chi2 distributions for each data sets}
\item {\color{red} TODO: plot the x or z dependece}
\item {\color{red} TODO: plot the $k_{\perp}$ or $P_{\perp}$ dependece}
\end{itemize}

\newpage
\section{Sivers}

\begin{itemize}
\item {\color{red} TODO: table of data sets with the following columns: exp, observable, num points, chi2 }
\item {\color{red} TODO: what TMDS are going to be fitted? what is the parametrization}
\item {\color{red} TODO: distribution of the parameters}
\item {\color{red} TODO: chi2 distributions for each data sets}
\item {\color{red} TODO: plot the x or z dependece}
\item {\color{red} TODO: plot the $k_{\perp}$ or $P_{\perp}$ dependece}
\end{itemize}

\newpage
\section{Collins function from $e^+e^-$}

\begin{itemize}
\item {\color{red} TODO: table of data sets with the following columns: exp, observable, num points, chi2 }
\item {\color{red} TODO: what TMDS are going to be fitted? what is the parametrization}
\item {\color{red} TODO: distribution of the parameters}
\item {\color{red} TODO: chi2 distributions for each data sets}
\item {\color{red} TODO: plot the x or z dependece}
\item {\color{red} TODO: plot the $k_{\perp}$ or $P_{\perp}$ dependece}
\end{itemize}

\newpage
\section{transversity}

\begin{itemize}
\item {\color{red} TODO: table of data sets with the following columns: exp, observable, num points, chi2 }
\item {\color{red} TODO: what TMDS are going to be fitted? what is the parametrization}
\item {\color{red} TODO: distribution of the parameters}
\item {\color{red} TODO: chi2 distributions for each data sets}
\item {\color{red} TODO: plot the x or z dependece}
\item {\color{red} TODO: plot the $k_{\perp}$ or $P_{\perp}$ dependece}
\end{itemize}

\newpage
\section{Boer-Mulders}

\begin{itemize}
\item {\color{red} TODO: table of data sets with the following columns: exp, observable, num points, chi2 }
\item {\color{red} TODO: what TMDS are going to be fitted? what is the parametrization}
\item {\color{red} TODO: distribution of the parameters}
\item {\color{red} TODO: chi2 distributions for each data sets}
\item {\color{red} TODO: plot the x or z dependece}
\item {\color{red} TODO: plot the $k_{\perp}$ or $P_{\perp}$ dependece}
\end{itemize}

\newpage
\section{pretzelosity}

\begin{itemize}
\item {\color{red} TODO: table of data sets with the following columns: exp, observable, num points, chi2 }
\item {\color{red} TODO: what TMDS are going to be fitted? what is the parametrization}
\item {\color{red} TODO: distribution of the parameters}
\item {\color{red} TODO: chi2 distributions for each data sets}
\item {\color{red} TODO: plot the x or z dependece}
\item {\color{red} TODO: plot the $k_{\perp}$ or $P_{\perp}$ dependece}
\end{itemize}

\newpage
\section{worm-gear}

\begin{itemize}
\item {\color{red} TODO: table of data sets with the following columns: exp, observable, num points, chi2 }
\item {\color{red} TODO: what TMDS are going to be fitted? what is the parametrization}
\item {\color{red} TODO: distribution of the parameters}
\item {\color{red} TODO: chi2 distributions for each data sets}
\item {\color{red} TODO: plot the x or z dependece}
\item {\color{red} TODO: plot the $k_{\perp}$ or $P_{\perp}$ dependece}
\end{itemize}



\newpage
\section{Monte Carlo Event Generator (MCEG)}

In this section we present details of how to construct an MCEG using
the analytic expression for the SIDIS cross sections given in
\eref{xsec}. The goal is to generate events for the reaction 
$$l+P\to l'+P_h+P_X$$
in the laboratory frame where $l$ and $l'$ are the momenta of the
incoming and outgoing leptons respectively, $P$ is the target
momentum, $P_h$ is the momentum of the observed hadron and $P_X$ is
unobserved system. In this frame, the initial momenta are given by  
%
\begin{align}
l&=(E,0,0,E)\notag\\
P&=(M,0,0,0)
\; ,
\end{align}
%
where $E$ is the incoming electron beam energy and $M$ is the target
nucleon mass. 

The SIDIS cross section in \eref{xsec} is defined in terms of
$x,y,z,\phi_h,\phi_S,P_h^T$. In order to generate the final state 
momenta, one needs to generate samples of  $x,y,z,\phi_h,\phi_S,P_h^T$
that are distributed according to \eref{xsec} and covert each sample
into the final state momenta $l',P_h$. Notice that $P_X$ is fixed by
momentum conservation. 

$\phi_h,\phi_S$ and $P_h^T$ in \eref{xsec} are defined in the Breit frame. To
transform the vectors into such frame we need to perform a rotation
$R$ followed by a Lorentz boost $B$. The rotation can be done by using the
Euler–Rodrigues formula by rotating $\vec{q}$  around the vector 
$\vec{u}=\vec{q}\times(-\hat{e}_z)$ 
by an angle $\alpha=\arccos(\vec{q}\cdot \hat{e}_z|\vec{q}|))$ with 
$\hat{e}_z=(0,0,1)$. The rotation of a any vector $\vec{x}$ is given by 
$$
R(\vec{x})=\vec{x}+2a(\vec{w}\times\vec{x})+2(\vec{w}\times(\vec{w}\times\vec{x}))
$$
where $a=\cos(\alpha/2)$ and $\vec{w}=\sin(\alpha/2)\vec{u}/|\vec{u}|$. The
boost $B$ is along the $z$-axis such that $B(R(q))=(0,0,0,-Q)$ and it
given by 
%
\begin{align}
B=\frac{1}{Q}
\begin{pmatrix} 
-q_R^3 & q_R^0 \\
q_R^0 & -q_R^3 
\end{pmatrix}
\end{align}
%
where $q_R=R(q)$.

We shall proceed to define a ``flat'' generator of the
phase space needed to be used in a given generic sampling algorithm.  
The generation of a phase space point proceeds with following 
steps:

\begin{enumerate}
\item Generate $x\in[1/(2ME),1]$, $Q2\in[1,2xME]$ and 
$\psi \in [0,2\pi]$ using uniform distributions and compute 
outgoing lepton momentum
%
\begin{align}
l'=E'\left(1,\sin(\theta)\cos(\phi),\sin(\theta)\sin(\phi),\cos(\theta)\right)
\, .
\end{align}
%
where $E'=E-Q^2/(2xM)$ and $\cos(\theta)=1-Q^2/(2 E E')$.
In addition compute $q=l-l'$.
\item Generate $z\in[M_h/q^0,1]$ using uniform distribution 
and compute the energy of the outgoing hadron $P_h^0=zq^0$
and the norm of outgoing hadron's 3-vector
$|\vec{P}_h|=\sqrt{P_h^0-M_h^2}$.
\item Generate $P_{h}^T\in[0,|\vec{P}_h|]$ and $\phi\in[0,2\pi]$
using uniform distributions and a random ${\rm sign} \in\{1,-1\}$
in order to generate outgoing hadron's momentum
%
\begin{align}
P_h=\left(P_h^0,
     P_{h}^T\cos(\phi),
     P_{h}^T\sin(\phi),
     {\rm sign}~\sqrt{|\vec{P}_h|^2-(P_h^T)^2}\right)
\end{align}
\item Rotate and boost all the 4-vectors $l,P,l',P_h,S$ as explained above
and proceed to compute the phase space parameters that enters in \eref{xsec}
\begin{align}
x&=\frac{Q^2}{2P\cdot q}\\
y&=\frac{P\cdot q}{P\cdot l}\\
z&=\frac{P\cdot P_h}{P\cdot q}\\
\phi_h&=\pi-\arccos((\vec{l}\times\vec{l'})\cdot (\vec{P_h}\times\vec{q}))\\
\phi_S&=\frac{\pi}{2}-\arccos((\vec{l}\times\vec{l'})\cdot S)\\
P_h^T &=\sqrt{(P_h^x)^2+(P_h^z)^2}
\end{align}
\end{enumerate}

The flat sampler outlined above can generate events for arbitrary
small $x,Q^2,z$ values. However, the analytic expressions for the 
SIDIS cross section in \eref{xsec} are only valid for $Q^2>1{~\rm GeV^2}$
to stay in the deep inelastic regime, $z>0.2$ to generate events 
in the current fragmentation region. The range of $x$ probed by the 
generator will depend on the beam energy $E$ and for fixed target 
experiments such as the Jefferson Lab 12 GeV the smallest value 
at $Q^2=1{~\rm GeV^2}$ is still in the domain of applicability 
of \eref{xsec}. The sampler can be embedded within a simple while loop 
that generate events and exit whenever the phase space parameters 
are within the desired range. 

{\color{red} 
Issues:

$\bullet$ The flat sampler does not produce flat $\phi_h$ and $\phi_S$ 

$\bullet$ Maybe we can to do uniform sampling on $\phi_h$ and $\phi_S$
and constrain azymuthal angles for $l'$ and $p_h$

}
 
%Having defined the flat sampler, we proceed to discuss the sampling 
%algorithms. 
%
%\subsection{Rejection Sampling}
%This is the simplest of the sampling algorithms. First compute the
%larges attainable value of the cross section within the phase space
%which occurs at  $Q^2_0=1{~\rm GeV^2}$, $x=Q^2_0/(2ME)$, $z=0.2$ and
%$P_h^T=0$. Let's denote a given phase space point as 
%$\Omega_i=(x_i,Q_i^2,z_i,(\phi_{h})_i,(P_{h}^T)_i)$. The rejection
%sampling proceed as 
%
%\begin{algorithm}
%  \caption{Rejection sampling}\label{rejection-sampling}
%  \begin{algorithmic}[1]
%%    \Procedure{Euclid}{$a,b$}\Comment{The g.c.d. of a and b}
%      \State EVENTS = empty list
%      \While{$r\not=0$}\Comment{We have the answer if r is 0}
%        \State $a\gets b$
%        \State $b\gets r$
%        \State $r\gets a\bmod b$
%      \EndWhile\label{euclidendwhile}
%      \For{\texttt{<some condition>}}
%        \State \texttt{<do stuff>}
%      \EndFor
%      \State \textbf{return} $b$\Comment{The gcd is b}
%%    \EndProcedure
%  \end{algorithmic}
%\end{algorithm}



\end{document}
