\documentclass[floatfix,aps,prd,nofootinbib,superscriptaddress,preprint]{revtex4}

\pdfoutput=1
\usepackage{graphicx}
\usepackage{amsmath}
\usepackage{amssymb}
\usepackage{slashed}
\usepackage{mathtools}
\usepackage[utf8]{inputenc}
\usepackage[colorlinks=true,linkcolor=blue,bookmarksopen,bookmarksnumbered]{hyperref}
\usepackage{color}
\usepackage[usenames,dvipsnames]{xcolor}
\usepackage[normalem]{ulem}
\usepackage{soul}
\usepackage{units}
\usepackage{rotating}
\usepackage{hhline,multirow,tabularx}
\usepackage{bm}

% math macros 
\newcommand{\T}[1]{\boldsymbol{#1}_{\text{T}}}
\newcommand{\kT}{\ensuremath{k_{\rm T}}}
\newcommand{\ktmax}{\ensuremath{k_{\rm T max}}}
\newcommand{\ktmaxsq}{\ensuremath{k_{\rm T max}^2}}
\newcommand\3[1]{\boldsymbol{#1}}
\newcommand{\Tsc}[2]{#1_{#2\text{T}}}
\newcommand{\Tscsq}[2]{#1^2_{#2\text{T}}}
\newcommand{\no}{\nonumber \\}
\newcommand{\parz}[1]{\ensuremath{\left(#1\right)}}
\newcommand{\order}[1]{\ensuremath{O\parz{#1}}}
\newcommand{\mhad}{\ensuremath{M}}
\newcommand{\alfa}{\ensuremath{\alpha}}
\newcommand{\xbj}{\ensuremath{x_{\rm bj}}}
\newcommand{\xn}{\ensuremath{x_{\rm n}}}
\newcommand{\mquark}{\ensuremath{m_{q}}}
\newcommand{\mgluon}{\ensuremath{m_{s}}}
\newcommand{\spectator}{\ensuremath{p_s}}
\newcommand{\jet}{\ensuremath{p_q}}
\newcommand{\pdfblob}{{\ensuremath{\mathcal F}}}
\newcommand{\ffblob}{{\ensuremath{\mathcal J}}}
\newcommand{\prop}{{[ \ensuremath{\rm Prop} ]}}
\newcommand{\num}{{\ensuremath{\rm T}}}
\newcommand{\jac}[1]{{\ensuremath{[ {\rm Jac_{#1}} ] }}}
\newcommand{\arrowcom}[1]{\textcolor{red}{\textbf{$\Longrightarrow$ #1}} \\}
\newcommand{\arrowcomtwo}[1]{\textcolor{blue}{ \textbf{$\Longrightarrow$ #1}} \\}
\newcommand{\arrowcomthree}[1]{\textcolor{magenta}{ \textbf{$\Longrightarrow$ #1}} \\}
\newcommand{\diff}[1]{\mathrm{d}#1}
\newcommand{\moderate}{{\color{red}low}}

\newcommand{\vect}[1]{\ensuremath{{\bm{#1}}}}
\newcommand{\bfkperp}{{\bf k}_\perp} 
\newcommand{\bfpperp}{{\bf P}_\perp} 
\newcommand{\bfhp}{\hat{\bf h}} 
\newcommand{\bfPhperp}{{\bf P}_{hT}}
\newcommand{\Phperp}{P_{hT}}
\newcommand{\kperp}{k_\perp}
\newcommand{\pperp}{P_\perp}
\newcommand{\avkperp}{\la \kperp^2 \ra}
\newcommand{\avpperp}{\la \pperp^2 \ra}
\newcommand{\la}{\langle}
\newcommand{\ra}{\rangle}


% ref macros
\newcommand{\eref}[1]{Eq.~(\ref{e.#1})}
\newcommand{\erefs}[2]{Eqs.~(\ref{e.#1})--(\ref{e.#2})}
\newcommand{\fref}[1]{Fig.~\ref{f.#1}}
\newcommand{\frefs}[2]{Figs.~\ref{f.#1}--\ref{f.#2}}
\newcommand{\aref}[1]{Appendix~\ref{a.#1}}
\newcommand{\sref}[1]{Sec.~\ref{s.#1}}
\newcommand{\ssref}[1]{Section~\ref{ss.#1}}
\newcommand{\sssref}[1]{Section~\ref{sss.#1}}
\newcommand{\tref}[1]{Table~\ref{t.#1}}








\begin{document}\title{Extraction and Validation Framework (EVA)}
{\color{red} TODO: update authors and affiliations}
\author{A.~Prokudin}
\author{D.~Riser}
\author{N.~Sato}
\author{K.~Tezgin}
\email{nsato@jlab.org}
\email{kemal.tezgin@uconn.edu}
\email{david.riser@uconn.edu}
\affiliation{University of Conneticut}

\maketitle

\newpage

\newcommand{\avp}{\ensuremath{\la P_{\perp}^2\ra_q}}
\newcommand{\avk}{\ensuremath{\la k_{\perp}^2\ra_q}}

\newcommand{\pdf}{{\color{blue} \ensuremath{f_1^q}}}
\newcommand{\ff}{{\color{red} \ensuremath{D_1^{q}}}}
\newcommand{\ppdf}{{\color{Emerald} \ensuremath{g_1^q}}}
\newcommand{\transversity}{{\color{gray} \ensuremath{h_1^q}}}
\newcommand{\sivers}{{\color{brown} \ensuremath{f^{\perp(1)q}_{1T}}}}
\newcommand{\boermulders}{{\color{purple} \ensuremath{h^{\perp(1)q}_1 }}}
\newcommand{\pretzelosity}{{\color{violet} \ensuremath{h^{\perp(1)q}_{1T} }}}
\newcommand{\collins}{{\color{orange} \ensuremath{H^{\perp(1)q}_{1} }}}
\newcommand{\gww}{{\color{magenta} \ensuremath{g^{\perp q}_{1T}}}}
\newcommand{\hww}{{\color{cyan}\ensuremath{h^{\perp q}_{1L}}}}

\newcommand{\wpdf}{0.57}
\newcommand{\wff}{0.12}
\newcommand{\wppdf}{0.19}
\newcommand{\wtransversity}{0.25}
\newcommand{\wsivers}{0.143}
\newcommand{\wboermulders}{0.085}
\newcommand{\wpretzelosity}{0.137}
\newcommand{\wcollins}{0.11}
\newcommand{\wgww}{0.19}
\newcommand{\whww}{0.25}


\section{SIDIS setup}

The differential cross section is given by 

\begin{align}
\frac{d\sigma}{dx ~ dy ~ d\Psi ~ dz ~ d\phi_h ~ d\Phperp^2} = 
\frac{\alpha^2}{xyQ^2}\frac{y^2}{2(1-\varepsilon)}\left(1+\frac{\gamma^2}{2x} \right) 
\sum_{i=1}^{18} F_i(x,z,Q^2,\Phperp^2) \beta_i 
\end{align}

\begin{table}[h!]
\begin{tabular}{|c|c|c|c|c|}
\hline
$F_i$    & Standard label                    & $\beta_i$\\\hline
$F_1$    & $F_{UU,T}$                        & 1\\\hline
$F_2$    & $F_{UU,L}$                        & $\varepsilon$\\\hline
$F_3$    & $F_{UU}^{\cos \phi_h}$            & $\sqrt{2\varepsilon (1+\varepsilon)} ~ \cos\phi_h$\\\hline
$F_4$    & $F_{UU}^{\cos 2\phi_h}$           & $\varepsilon ~ \cos(2\phi_h)$\\\hline
$F_5$    & $F_{LU}^{\sin\phi_h}$             & $\lambda_e\sqrt{2\varepsilon (1-\varepsilon)} \sin\phi_h $\\\hline
$F_6$    & $F_{UL}^{\sin\phi_h}$             & $S_{||}\sqrt{2\varepsilon (1+\varepsilon)}\sin\phi_h$\\\hline
$F_7$    & $F_{UL}^{\sin ~ 2\phi_h}$         & $S_{||}\varepsilon \sin(2\phi_h)$\\\hline
$F_8$    & $F_{LL}$                          & $S_{||} \lambda_e \sqrt{1-\varepsilon^2}$\\\hline
$F_9$    & $F_{LL}^{\cos\phi_h}$             & $S_{||} \lambda_e  \sqrt{2\varepsilon(1-\varepsilon)} \cos\phi_h$\\\hline
$F_{10}$ & $F_{UT,T}^{\sin(\phi_h-\phi_S)}$  & $|\vec{S}_\perp|\sin(\phi_h-\phi_S)$\\\hline
$F_{11}$ & $F_{UT,L}^{\sin(\phi_h-\phi_S)}$  & $|\vec{S}_\perp|\varepsilon\sin(\phi_h-\phi_S)$\\\hline
$F_{12}$ & $F_{UT}^{\sin(\phi_h+\phi_S)}$    & $|\vec{S}_\perp|\varepsilon\sin(\phi_h+\phi_S)$\\\hline
$F_{13}$ & $F_{UT}^{\sin(3\phi_h - \psi_S)}$ & $|\vec{S}_\perp| \varepsilon \sin(3\phi_h - \phi_S)$\\\hline
$F_{14}$ & $F_{UT}^{\sin\phi_S}$             & $|\vec{S}_\perp| \sqrt{2\varepsilon (1+\varepsilon)}\sin\phi_S$\\\hline
$F_{15}$ & $F_{UT}^{\sin(2\phi_h-\phi_S)}$   & $|\vec{S}_\perp| \sqrt{2\varepsilon (1+\varepsilon)}\sin(2\phi_h-\phi_S)$\\\hline
$F_{16}$ & $F_{LT}^{\cos(\phi_h-\phi_S)}$    & $|\vec{S}_\perp|\lambda_e \sqrt{1-\varepsilon^2}\cos(\phi_h-\phi_S)$\\\hline
$F_{17}$ & $F_{LT}^{ \cos\phi_S}$            & $|\vec{S}_\perp|\lambda_e \sqrt{2\varepsilon(1-\varepsilon)} \cos\phi_S$\\\hline
$F_{18}$ & $F_{LT}^{\cos(2\phi_h-\phi_S)}$   & $|\vec{S}_\perp|\lambda_e \sqrt{2\varepsilon(1-\varepsilon)} \cos(2\phi_h-\phi_S)$\\\hline
\end{tabular}
\end{table}

\begin{itemize}
\item {\color{red} TODO: what is $\Psi$?}
\item {\color{red} TODO: explicit expression for $|S_{\perp}|$}
\item {\color{red} TODO: explicit expression for $\varepsilon$}
\end{itemize}

The 18 structure function in SIDIS at leading-order will be expressed
in the context of WW-type approximation in terms of a ``minimal''
TMD basis using the gaussian ansatz:
%
\begin{align}
{\cal F}_q(\xi,p_{\perp})&={\cal K}_q ~ {\cal C}_q(\xi) \frac{\exp\left(-k_{\perp}^2/\omega_q\right)}{\pi \omega_q}\\
{\cal D}_q(\xi,p_{\perp})&={\cal K}_q ~ {\cal C}_q(\xi) \frac{\exp\left(-P_{\perp}^2/\omega_q\right)}{\pi \omega_q}.
\end{align}
%
We denote the transverse momentum of the quark inside a fast moving
proton by $\bfkperp$. We use the notation $\bfpperp$ for the
transverse momentum of the quark relative to the original parton
motion. The structure functions are expressed as
%
\begin{align}
F&=\sum_q e_q^2 ~ {\cal K}_q ~ {\cal F}_q(x) ~{\cal D}_q(z) \frac{\exp\left(-\Phperp^2/\Omega_q\right)}{\pi \Omega_q}\\
\Omega_q&=z^2\avk + \avp
\end{align}

%

\begin{table}[h!]
\begin{tabular}{|c|c|c|c|c|}
\hline
type       & Name         & ${\cal K}_q$                  & ${\cal C}_q$    \\\hline
${\cal F}_q$ & upol. PDF    & $1$                           & $\pdf$          \\\hline
${\cal F}_q$ & pol. PDF     & $1$                           & $\ppdf$         \\\hline
${\cal F}_q$ & Transversity & $1$                           & $\transversity$ \\\hline
${\cal F}_q$ & Sivers       & $\frac{2M^2}{\omega_q}$       & $\sivers$       \\\hline
${\cal F}_q$ & Boer-Mulders & $\frac{2M^2}{\omega_q}$       & $\boermulders$  \\\hline
${\cal F}_q$ & Pretzelosity & $\frac{2M^2}{\omega_q}$       & $\pretzelosity$ \\\hline
${\cal F}_q$ & Worm Gear    & $1$                           & $\gww$          \\\hline
${\cal F}_q$ & Worm Gear    & $1$                           & $\hww$          \\\hline
\hline
${\cal C}_q$ & FF           & $1$                           & $\ff$           \\\hline
${\cal C}_q$ & Collins      & $\frac{2z^2 m_h^2}{\omega_q}$ & $\collins$      \\\hline
\end{tabular}
\caption{The minimal basis}
\label{t.mbasis}
\end{table}


\begin{table}[h!]
\begin{tabular}{|c|c|c|c|c|c|c|}
\hline
         &                                     & ${\cal{K}}_q$                                                                               & ${\cal F}_q(x)$ & ${\cal D}_q(z)$ \\\hline
$F_1$    & $F_{UU,T}$                          & $x$                                                                                         & $\pdf$          & $\ff $          \\\hline
$F_2$    & $F_{UU,L}$                          & $0$                                                                                         &                 & \\\hline
$F_3$    & $F_{LL}$                            & $x$                                                                                         & $\ppdf$         & $\ff$           \\\hline
$F_4$    & $F_{UT}^{\sin(\phi_h+\phi_S)}$      & $\frac{2 x z \Phperp  m_h}{w_q}$                                                            & $\transversity$ & $\collins$      \\\hline
$F_5$    & $F_{UT,T}^{\sin(\phi_h-\phi_S)}$    & $-\frac{2xzM \Phperp}{w_q}$                                                                 & $\sivers$       & $\ff$           \\\hline
$F_6$    & $F_{UT,L}^{\sin(\phi_h-\phi_S)}$    & $0$                                                                                         &                 & \\\hline
$F_7$    & $F_{UU}^{\cos(2\phi_h)}$            & $\frac{4 x z^2 M \Phperp^2 m_h}{w_q^2} $                                                    & $\boermulders$  & $\collins $     \\\hline
$F_8$    & $F_{UT}^{\sin(3\phi_h-\phi_S)}$     & $\frac{2 x z^3 \Phperp^3 m_h \avk}{w_q^3} $                                                 & $\pretzelosity$ & $\collins $     \\\hline
$F_9$    & $F_{LT}^{\cos(\phi_h -\phi_S)}$     & $ \frac{2xzM\Phperp}{w_q} $                                                                 & $\gww$          & $\ff$           \\\hline
$F_{10}$ & $F_{UL}^{\sin(2\phi_h)}$            & $ \frac{4 xz^2M \Phperp^2 m_h}{w_q^2} $                                                     & $\hww$          & $\collins $     \\\hline
$F_{11}$ & $F_{LT}^{\cos\phi_S}$               & $-\frac{2M}{Q} x\frac{z^2\avk \left[\Phperp^2+ \avp\right] + \avpperp^2}{ w_q^2}$           & $\gww$          & $\ff$           \\\hline
$F_{12}$ & $F_{LL}^{\cos\phi_h}$               & $-\frac{2xz\Phperp}{Q}\frac{\avk}{w_q}$                                                     & $\ppdf$         & $\ff$           \\\hline
$F_{13}$ & $F_{LT}^{\cos(2\phi_h -\phi_S)}$    & $-\frac{2 x z^2 M \Phperp^2}{Q}\frac{\avk}{w_q^2}$                                          & $\gww$          & $\ff$           \\\hline
$F_{14}$ & $F_{UL}^{\sin\phi_h}$               & $-\frac{8M^3}{Q} x \frac{z^2\avk (\Phperp^2-z^2\avk)+\avp^2}{w_q^3}$                        & $\hww$          & $\collins$      \\\hline
$F_{15}$ & $F_{LU}^{\sin\phi_h}$               & $0$                                                                                         &                 & \\\hline
$F_{16}$ & $F_{UU}^{\cos\phi_h}(i)$            & $-\frac{8M}{Q} xz\Phperp m_h \frac{\left[\avp^2+z^2 \avk (\Phperp^2  -z^2\avk ) \right]}{w_q^3}$ & $\boermulders$  & $\collins$      \\\hline
$F_{16}$ & $F_{UU}^{\cos\phi_h}(ii)$           & $-\frac{2M}{Q} \frac{xz\Phperp}{M} \frac{\avk}{w_q}$                                                           & $\pdf$          & $\ff $          \\\hline
$F_{17}$ & $F_{UT}^{\sin\phi_S}(i)$            & $-\frac{2M}{Q} x \frac{z^2\avk \left(\Phperp^2+\avp\right) + \avp^2}{w_q^2}$                & $\sivers$       & $\ff$           \\\hline
$F_{17}$ & $F_{UT}^{\sin\phi_S}(ii)$           & $\frac{4 x z^2 m_h}{Q}\frac{\avk\left(-\Phperp^2+w_q\right)}{w_q^2} $                       & $\transversity$ & $\collins$      \\\hline
$F_{18}$ & $F_{UT}^{\sin(2\phi_h-\phi_S)}(i)$  & $-\frac{2M^2}{Q} x \frac{\avk M}{w_q^2} $                                                   & $\sivers$       & $\ff$           \\\hline
$F_{18}$ & $F_{UT}^{\sin(2\phi_h-\phi_S)}(ii)$ & $-\frac{2M^2}{Q} x \frac{4z^2 \Phperp^2 m_h}{w_q^2}$                                        & $\pretzelosity$ & $\collins$      \\\hline
\end{tabular}
\caption{SIDIS structure functions up to twist 3}
\label{t.chi2}
\end{table}



%with the following parametrizations 
%
%\begin{align}
%  \transversity &= \frac{1}{2} ~ N_q(x) \left[ \pdf + \ppdf \right] \nonumber \\
%  \sivers &= -\sqrt{\frac{e}{2}} ~ N_q(x) \frac{1}{M ~ M_S} \frac{\Omega_5^2}{\Omega_1} \pdf \nonumber \\
%  \boermulders &= -\sqrt{\frac{e}{2}} ~ N_q(x) \frac{1}{M ~ M_{BM}} \frac{\Omega_6^2}{\Omega_1} \pdf \nonumber \\
%  \pretzelosity &= e ~ N_q(x) \frac{1}{M_P^2} \frac{\Omega_7^2}{\Omega_1} \left[ \pdf - \ppdf \right] \nonumber \\
%  \collins &= \sqrt{\frac{e}{2}} ~ N_q(x) \frac{M_C^3}{z ~ m_h} \frac{\Omega_8}{(M_C^2 + \Omega_8)^2} \ff \nonumber \\ 
%  \gww &= x\int_x^1 \frac{dy}{y}\ppdf(y) \nonumber \\
%  \hww &= x\int_x^1 \frac{dy}{y}\transversity(y)
%\end{align}
%where the PDFs $\pdf$,$\ppdf$,$\ff$ are experimentally accessible and are obtained from CJ15, LSS and DSS libraries respectively. The $WW$-type approximations are used to obtain the  PDFs $\gww$ and $\hww$.  


\newpage
\section{SIA setup}

{\color{red} TODO: add table of SIA structure functions}

\newpage
\section{Tutorial}
{\color{red} TODO: Add more details }

Here is a general workflow

\begin{itemize}
\item Identify which parameters are needed to be fitted for a given set of TMDs
\item Identify available observables to extract the TMDs 
\item Create xlsx files for each data sets (see the database repo)
\item Create and input file (see i.e. fitlab/inputs/upol.py). Here you will specify 
      which data sets are going to be used
\item Test in a single fit (fitter.py) that a reasonable $\chi^2$ is obtained 
\item Once the test fit is ready, proceed to run a multinest which is an nested sampling algorithm 
      to map out the likelihood function. In the input file you must specify the output.
      Look for the line "output dir " in the input file. It is recommended to store the results in the 
      repo analysis. 
\item Once the multinest finishes it produces a nestout file at the path that was specified. 
      Use one of the jupyter templates to proceed to analyze the output. You need to place the notebook 
      at the root of analysis. Otherwise the paths wont match.  
\end{itemize}

If you want to run the codes but you want to modify things here and there, 
create a new workspace via pacman/gen-fitpack, and cd to fitpack. The latter 
has an exact copy of all the repos, and you can do with that whatever you want. 
Once you see that some of your modifications should be placed in the actual
repositories, proceed to do it.

\newpage
\section{Unpolarized TMDS}

\begin{itemize}
\item {\color{red} TODO: table of data sets with the following columns: exp, observable, num points, chi2 }
\item {\color{red} TODO: what TMDS are going to be fitted? what is the parametrization}
\item {\color{red} TODO: distribution of the parameters}
\item {\color{red} TODO: chi2 distributions for each data sets}
\item {\color{red} TODO: plot the x or z dependece}
\item {\color{red} TODO: plot the $k_{\perp}$ or $P_{\perp}$ dependece}
\end{itemize}

\newpage
\section{Sivers}

\begin{itemize}
\item {\color{red} TODO: table of data sets with the following columns: exp, observable, num points, chi2 }
\item {\color{red} TODO: what TMDS are going to be fitted? what is the parametrization}
\item {\color{red} TODO: distribution of the parameters}
\item {\color{red} TODO: chi2 distributions for each data sets}
\item {\color{red} TODO: plot the x or z dependece}
\item {\color{red} TODO: plot the $k_{\perp}$ or $P_{\perp}$ dependece}
\end{itemize}

\newpage
\section{Collins function from $e^+e^-$}

\begin{itemize}
\item {\color{red} TODO: table of data sets with the following columns: exp, observable, num points, chi2 }
\item {\color{red} TODO: what TMDS are going to be fitted? what is the parametrization}
\item {\color{red} TODO: distribution of the parameters}
\item {\color{red} TODO: chi2 distributions for each data sets}
\item {\color{red} TODO: plot the x or z dependece}
\item {\color{red} TODO: plot the $k_{\perp}$ or $P_{\perp}$ dependece}
\end{itemize}

\newpage
\section{transversity}

\begin{itemize}
\item {\color{red} TODO: table of data sets with the following columns: exp, observable, num points, chi2 }
\item {\color{red} TODO: what TMDS are going to be fitted? what is the parametrization}
\item {\color{red} TODO: distribution of the parameters}
\item {\color{red} TODO: chi2 distributions for each data sets}
\item {\color{red} TODO: plot the x or z dependece}
\item {\color{red} TODO: plot the $k_{\perp}$ or $P_{\perp}$ dependece}
\end{itemize}

\newpage
\section{Boer-Mulders}

\begin{itemize}
\item {\color{red} TODO: table of data sets with the following columns: exp, observable, num points, chi2 }
\item {\color{red} TODO: what TMDS are going to be fitted? what is the parametrization}
\item {\color{red} TODO: distribution of the parameters}
\item {\color{red} TODO: chi2 distributions for each data sets}
\item {\color{red} TODO: plot the x or z dependece}
\item {\color{red} TODO: plot the $k_{\perp}$ or $P_{\perp}$ dependece}
\end{itemize}

\newpage
\section{pretzelosity}

\begin{itemize}
\item {\color{red} TODO: table of data sets with the following columns: exp, observable, num points, chi2 }
\item {\color{red} TODO: what TMDS are going to be fitted? what is the parametrization}
\item {\color{red} TODO: distribution of the parameters}
\item {\color{red} TODO: chi2 distributions for each data sets}
\item {\color{red} TODO: plot the x or z dependece}
\item {\color{red} TODO: plot the $k_{\perp}$ or $P_{\perp}$ dependece}
\end{itemize}

\newpage
\section{worm-gear}

\begin{itemize}
\item {\color{red} TODO: table of data sets with the following columns: exp, observable, num points, chi2 }
\item {\color{red} TODO: what TMDS are going to be fitted? what is the parametrization}
\item {\color{red} TODO: distribution of the parameters}
\item {\color{red} TODO: chi2 distributions for each data sets}
\item {\color{red} TODO: plot the x or z dependece}
\item {\color{red} TODO: plot the $k_{\perp}$ or $P_{\perp}$ dependece}
\end{itemize}

\end{document}
