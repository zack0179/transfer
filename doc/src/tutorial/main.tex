%%%%%%%%%%%%%%%%%%%%%%%%%%%%%%%%%%%%%%%%%%%%%%%%%%%%%%%%%%%%%
\begin{frame}
\frametitle{\textbf{Installation}}
%------------------------------------------------------------
\begin{textblock}{110}(5,12) 

\begin{itemize}
\item \textbf{step 1:}
  \begin{itemize}
  \item[+] Install python2 scipy stack. e.g. anaconda (python2) 
  \item[+] From the commandline:  
     \begin{itemize}
     \item[] \textbf{\texttt{cd external \&\& ./install}}
     \end{itemize}
  \end{itemize}

\item \textbf{step 2:}
  \begin{itemize}
     \item[+] Enviroment variables needs to be set in your
     terminal session. 
     \item[+] For \texttt{bash} they are
        \begin{itemize}
        \item[] \textbf{\texttt{export FITPACK=path2fitpack}}
        \item[] \textbf{\texttt{PYTHONPATH=\$FITPACK:\$PYTHONPATH}}
        \item[] \textbf{\texttt{export PATH=\$FITPACK/bin:\$PATH}}
        \end{itemize}
     \item[+] Alternatively you can source the setup files:
        \begin{itemize}
        \item[] \textbf{\texttt{source  setup.bash}}
        \end{itemize}
  \end{itemize}
\end{itemize}

\end{textblock}
%------------------------------------------------------------
\end{frame}

%%%%%%%%%%%%%%%%%%%%%%%%%%%%%%%%%%%%%%%%%%%%%%%%%%%%%%%%%%%%%
\begin{frame}
\frametitle{\textbf{Workflow}}
%------------------------------------------------------------
\begin{textblock}{110}(5,12) 
\begin{itemize}
\item \textbf{General}
      \begin{itemize}
      \item[+] There are various ways in which  \texttt{fitpack} can be used 
      \item[+] The enviroment variables allows you to run the code
               from anywhere in your system. In other words, there is no need 
               to work within the same folder of \texttt{fitpack}
      \item[+] As a good practice, create dedicated folders for a
               given analysis you want to study 
      \end{itemize}
\item \textbf{Using the terminal }
      \begin{itemize}
      \item[+] There are dedicated scripts that can be executed from
               commandline. They are located at \texttt{fitpack/bin/}
      \item[+] The main script is called \texttt{jam3d}. This is
               useful for running the \texttt{jam3d} codes to generate TMD MC 
               parameters via nested sampling. The code can be
               packaged into  containers and be deployed in cluster 
               environments. 
      \end{itemize}

\end{itemize}

\end{textblock}
%------------------------------------------------------------
\end{frame}


%%%%%%%%%%%%%%%%%%%%%%%%%%%%%%%%%%%%%%%%%%%%%%%%%%%%%%%%%%%%%
\begin{frame}
\frametitle{\textbf{Workflow}}
%------------------------------------------------------------
\begin{textblock}{110}(5,12) 

\begin{itemize}
\item \textbf{Using the Jupyer notebooks}
      \begin{itemize}
      \item[+] The core libraries can also be loaded from jupyer notebooks
      \item[+] The notebooks are more useful for visualization task
               such as plotting data or TMDs. 
      \item[+] This is ideal to share the software via \texttt{jupter-hub}
               servers. 
      \end{itemize}

\item \textbf{Terminal or jupyter }
      \begin{itemize}
      \item[+] In principle all the workflow can be set inside a
               jupyter notebook without ever need to run programs 
               from the commandline. 
      \item[+] However at the beginning of any analysis where one
               needs to know if a particular implementation of the
               theory works by fitting the data, it is simpler to check  
               from commandline if the setup works
      \item[+] Moreover, for more complex problem where many parameters 
               are involved in the analysis or a MC sampling is
               desired, is best to proceed via terminal (specially for
               very long runs)and use  jupyter-notebooks to post-process the results
      \end{itemize}
\end{itemize}

\end{textblock}
%------------------------------------------------------------
\end{frame}


%%%%%%%%%%%%%%%%%%%%%%%%%%%%%%%%%%%%%%%%%%%%%%%%%%%%%%%%%%%%%
\begin{frame}
\frametitle{\textbf{Tutorial 1: fits of unpolarized TMDs}}
%------------------------------------------------------------
\begin{textblock}{110}(5,12) 

\begin{itemize}
\item \textbf{Using the Jupyer notebooks}
  \begin{itemize}
  \item[+] The core libraries can also be loaded from jupyer notebooks
  \item[+] The notebooks are more useful for visualization task
           such as plotting data or TMDs. 
  \item[+] This is ideal to share the software via \texttt{jupter-hub}
           servers. 
  \end{itemize}

\end{itemize}

\end{textblock}
%------------------------------------------------------------
\end{frame}























